% !TEX encoding = UTF-8 Unicode
\documentclass[a4paper]{article}


\usepackage[T1]{fontenc}     % För svenska bokstäver
\usepackage[utf8]{inputenc}  % Teckenkodning UTF8
\usepackage[swedish]{babel}  % För svensk avstavning och svenska
                             % rubriker (t ex "Innehållsförteckning")
\usepackage{fancyvrb}        % För programlistor med tabulatorer
\fvset{tabsize=4}            % Tabulatorpositioner
\fvset{fontsize=\small}      % Lagom storlek för programlistor

\title{Mandelbrot \\
	Inlämningsuppgift 2, Programmeringsteknik för C/D}
\author{Adam Hansson Lyrén, C11 (dic11aha@student.lu.se)\\
Johan Bäversjö, C11 (dic11jba@student.lu.se)}


% *** Tillägg för denna rapport. ***
% Paket:
\usepackage{graphicx}         % För att inkludera bilder.

% Kommandon i denna rapport
\newcommand{\code}[1]{\texttt{#1}} % För programkod i text.
% *** Slut på tillägg för denna rapport. ***


\begin{document}              % Början på dokumentet

\maketitle
\thispagestyle{empty}
\newpage
\setcounter{page}{1}
\section{Bakgrund}
Uppgiften består i att skapa en Mandelbrots-figur. Vi ska till en börjar skapa en klass som hanterar komplexa tal och därigenom skapa ett komplext tal-plan som i sin tur skall efter olika beräkningar, bilda Mandelbrotsfiguren. Vi har fått ett färdigt UI att jobba med där vi behöver implementera lite olika funktioner som: zoomning, återställning av figuren och målning figuren.
\newline
Vi skall dessutom ge stöd till inställningar: med eller utan färg, upplösning och ett fält med extra inställningar som kan vara lite vad som helst.
\newline
\newline
Så här kommer en Mandelbrotsfigur att se ut:
\begin{center}
\includegraphics[scale=0.29]{mandelbrot_print_1.jpg}
\end{center}
 
\section{Modell}
Modellen av programmet innehåller följande klasser:


\begin{tabular}{lp{8cm}}
\code{Complex} & Klass som beskriver ett Komplext tal i planet \\
\code{TestComplex} & Klass som beskriver ett unit-test för klassen \code{Complex} \\
\code{Mandelbrot} & Innehåller \code{main} metoden som skapar det färdigställda GUI:t och hanterar kommandon \\
\code{Generator} & Innehåller ett antal metoder som genererar det komplexa tal-planet och målar ut en figur beroende på om talet är inom Mandelbrots-mängden eller inte \\

\end{tabular}

\vspace{\baselineskip}
Loppet implementeras genom att programmet itererar tills en av sköldpaddorna nått mållinjen.
Nedan följer en beskrivning till flödet i programmet:

\begin{itemize}
\item Fönstret, banan och sköldpaddorna initieras. Banan ritas ut i fönstret av typ \code{SimpleWindow} i klassen \code{RaceTrack}.
\item Ett lopp skapas med klassen \code{RacingEvent}
\item Loppet initieras i och med ett musklick i fönstret. Då ställs sköldpaddorna först upp på startpositionerna, som hämtas från \code{RaceTrack},  i metoden \code{race()} från klassen \code{RacingEvent}
\item Metoden \code{race()} returnerar när en av sköldpaddorna nått mållinjen, därefter är programmet färdigt. 
\end{itemize}

\section{Brister och kommentarer}
Istället för att användaren manuellt behöver specifiera antalet iterationer, som behövs göras vid mycket inzoomning hade programmet kunna sköta detta. Antalet iterationer skulle kunna vara en funktion av zoom nivån för att alltid ge en högupplöst bild.

De färger som används i färgläget hade användaren själv kunnat specifiera med en färgväljare i gränssnittet. Man hade kunnat välja ett obegränsat antal färger och se direkt hur de blandats i ett spektrum i GUI:t.

För att kunna koppla ihop det här programmet med andra system hade vi velat att programmet skulle ha ett CLI. Detta hade gjort det enkelt att koppla ihop vårt program med andra system. Programmet skulle kunna ta fokuspunkt, zoom och antal iterationer som inparametrar och en JPG bild som output. Detta hade varit ett enkelt tilläg till programmet då det koden inte behövs modifieras mycket. En ny GUI implementation som emulerar ett gui hade behövts skrivas, samt kod för att hantera input/output.


\section{Programlistor}
Klasserna finns i filer med motsvarande namn. Till exempel innehåller filen  \code{Generator.java} klassen \code{Generator}. Alla klasser som används finns i samma katalog som huvudprogrammet

\subsection{\code{Complex}}
\VerbatimInput{../src/Complex.java}

\subsection{\code{TestComplex}}
\VerbatimInput{../src/TestComplex.java}

\subsection{\code{Mandelbrot}}
\VerbatimInput{../src/Mandelbrot.java}

\subsection{\code{Generator}}
\VerbatimInput{../src/Generator.java}

\end{document}                  % Slut på dokumentet
